
\begin{enumerate}
	\item Алеаторная и эпистемическая неопределенности. Выборка, генеральная совокупность, случайная выборка, эмпирическое распределение, эмпирическая случайная величина.
	\item Сходимость эмпирической функции распределения почти наверное и по Колмогорову. Plug-in оценка.
	\item Характеристика положения. Примеры. Характеристика разброса. Примеры. Асимметрия, разложение Эджворта.
	\item Выборочное среднее, выборочная дисперсия, исправленная дисперсия, выборочное стандартное отклонение, исправленное стандартное отклонение, выборочная асимметрия. Вариационный ряд, $k$-я порядковая статистика. Выборочный квантиль, выборочная медиана.
	\item Статистика, оценка. Plug-in оценка. Смещение, разброс, среднеквадратичная ошибка(MSE), RMSE. Пример для выборочного среднего и выборочной дисперсии.
	\item Асимптотическая несмещенность. Пример. Состоятельная оценка. Пример. 
	\item Асимптотическая нормальность и коэффициент рассеивания. Из асимптотической нормальности следует состоятельность.
	\item Два критерия состоятельности оценки. Теорема о асимптотической нормальности от дифференцируемой функции.. Дельта метод.
	\item Бутстрап. Оценка смещения. Оценка разброса. Виды ошибок бутстрапа.
	\item Доверительный интервал: асимптотический, точный. Центральный, левый, правый. Качества интервала: точность, порядок.
	\item Модели: параметрические, непараметрические, семипараметрические. 
	\item Распределение Бернулли, биномиальное, геометрическое, отрицательно биномиальное.
	\item Распределение равномерное, экспоненциальное, Гамма, Пуассона.
	\item Нормальное распределение.
	\item Метод моментов. Функция и метод правдоподобия. Пример для биномиального и нормального распределений.
	\item Информационная энтропия. Расстояние Кульбака-Лейблера. Состоятельность ОМП.

	\item Носитель. Регулярная модель. Перестановка интеграла и дифференцирования для регулярной модели. Sanity check для градиента.
	\item Асимптотическая нормальность ОМП на пальцах. Пример для распределения Бернулли и нормального распределения.
	\item Распределения, связанные с нормальным(хи-квадрат, Стьюдента, Фишера).
	\item Normal theory: интервал для среднего. Случай когда $\sigma^2$ неизвестно, лемма Фишера.
	\item Метод центральной функции.
	\item Стабилизация дисперсии.
	\item Эфронов интервал. Алгоритм построения интервала для параметра и статистики.

	\item M-оценка. Примеры. Асимптотическая нормальность М-оценок. Пример для линейной регрессии и ОМП для $\mathcal P_X$ не из модели.
	\item Эффективность. Асимптотическая эффективность. Теорема Рио-Крамера, док-во для одномерного случая. Пример для $\mathcal N(\mu, \sigma^2)$, где $\mu$ неизвестно.
	\item Условное математическое ожидание. Свойства. Достаточная статистика. Теорема факторизации. Пример Бернуллиевского и нормального распределений.
	\item Теорема Колмогорова-Блэкуэлла-Рао. Пример для Бернуллиевского распределения со статистикой $\phi = p, \phi^*= x_1, T = \sum x_i$.
	\item Полная статистика. Теорема Лемана-Шеффе. Пример для статистика $x_{(n)}$ в $U([0, \theta])$.
	\item Функция выживаемости. Тяжелые и легкие хвосты. Распределение Парето. Принцип катастроф и принцип заговора. Функция риска.
	\item Робастность. Пороговая точка. Huber estimator. Базовые методы робастного оценивания.
	
	\item Проверка гипотез: согласия, однородности, о значении характеристики. Нулевая гипотеза, Альтернатива. Критерий, статистика критерия, критическое множество/семейство. P-value. Алгоритм проверки гипотез через значимость.
	\item Размер эффекта и minimum detectable effect(MDE). Проверка гипотез через размер эффекта.
	\item Ошибки первого и второго родов, мощность, точность. Асимптотичность и состоятельность критерия, наиболее мощных критерий.

	\item Простая гипотеза и простая альтернатива, рандомизированный критерий. Критерий Неймана—Пирсона. Пример для $X \sim \mathcal N(\mu \in \R, \sigma^2), H_0 \colon \mu = \mu_0, \, H_1 \colon \mu = \mu_1 > \mu_0$. Минимальный объем выборки.
	\item Sequential Probability Ratio Test. Теорема Вальда.
	\item Сложный гипотезы, ошибки I/II родов. Гипотезы о характеристиках(аналогично центральным функциям). Z-тест. T-тест. Скорректированный T-тест.

	\item A/B тестирование. Целевая метрика, истинный эффект.  Z-тест для матожиданий двух выборок. Обобщение: Z-тест для матожиданий двух выборок, Z-тест для пропорций, T-тест для матожиданий двух выборок, T-тест Уэлча.
	\item Бутстрап. Критерий Манна—Уитни. ROC-AUC.
	\item Вариабельность и маленький эффект. Повторные наблюдения. Необходимый объем выборки для $\alpha = 0.05, \beta = 0.2$. M-ошибка.

	%11 лекция
	\item Валидация модели. Q-Q plot. Гипотезы согласия. Критерий Колмогорова, в том числе для сложной $H_0$. Критерий Лиллиефорса. Критерии типа $\omega^2$.
	\item Критерий согласия $\chi^2$ для простой и сложной гипотез.

	
	
\end{enumerate}
