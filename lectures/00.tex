
\begin{enumerate}
	\item Алеаторная и эпистемическая неопределенности. Выборка, генеральная совокупность, случайная выборка, эмпирическое распределение, эмпирическая случайная величина.
	\item Сходимость эмпирической функции распределения почти наверное и по Колмогорову. Plug-in оценка.
	\item Характеристика положения. Примеры. Характеристика разброса. Примеры. Асимметрия, разложение Эджворта.
	\item Выборочное среднее, выборочная дисперсия, исправленная дисперсия, выборочное стандартное отклонение, исправленное стандартное отклонение, выборочная асимметрия. Вариационный ряд, $k$-я порядковая статистика. Выборочный квантиль, выборочная медиана.
	\item Статистика, оценка. Plug-in оценка. Смещение, разброс, среднеквадратичная ошибка(MSE), RMSE. Пример для выборочного среднего и выборочной дисперсии.
	\item Асимптотическая несмещенность. Пример. Состоятельная оценка. Пример. 
	\item Асимптотическая нормальность и коэффициент рассеивания. Из асимптотической нормальности следует состоятельность.
	\item Два критерия состоятельности оценки. Теорема о асимптотической нормальности от дифференцируемой функции.. Дельта метод.
	\item Бутстрап. Оценка смещения. Оценка разброса. Виды ошибок бутстрапа.
	\item Доверительный интервал: асимптотический, точный. Центральный, левый, правый. Качества интервала: точность, порядок.
	\item Модели: параметрические, непараметрические, семипараметрические. 
	\item Распределение Бернулли, биномиальное, геометрическое, отрицательно биномиальное.
	\item Распределение равномерное, экспоненциальное, Гамма, Пуассона.
	\item Нормальное распределение.
	\item Метод моментов. Функция и метод правдоподобия. Пример для биномиального и нормального распределений.
	\item Информационная энтропия. Расстояние Кульбака-Лейблера. Состоятельность ОМП.

	\item Носитель. Регулярная модель. Sanity check для градиента.
	\item Асимптотическая нормальность ОМП на пальцах. Пример для распределения Бернулли и нормального распределения.
	\item Распределения, связанные с нормальным(хи-квадрат, Стьюдента, Фишера).
	\item Normal theory: интервал для среднего. Случай когда $\sigma^2$ неизвестно, лемма Фишера.
	\item Метод центральной функции.
	\item Стабилизация дисперсии.
	\item Эфронов интервал. Алгоритм построения интервала для параметра и статистики.
	
\end{enumerate}
